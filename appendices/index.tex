\chapter{Índice}

Este capítulo contiene el índice completo del libro "Think Python 2", que sirve como referencia rápida para los términos y conceptos clave cubiertos en el texto.

\section{Índice alfabético}

\begin{description}
\item[abecedarian] 73, 84
\item[abs function] 52
\item[absolute path] 139, 145
\item[access] 90
\item[accumulator] 100
  \begin{itemize}
  \item histogram 127
  \item list 93
  \item string 175
  \item sum 93
  \end{itemize}
\item[Ackermann function] 61, 113
\item[add method] 165
\item[addition with carrying] 68
\item[algorithm] 67, 69, 130, 201
  \begin{itemize}
  \item MD5 146
  \item square root 69
  \end{itemize}
\item[aliasing] 95, 96, 100, 149, 151, 170
  \begin{itemize}
  \item copying to avoid 99
  \end{itemize}
\item[all] 186
\item[alphabet] 37
\item[alternative execution] 41
\item[ambiguity] 5
\item[anagram] 101
\item[anagram set] 123, 145
\item[analysis of algorithms] 201, 209
\item[analysis of primitives] 204
\item[and operator] 40
\item[any] 185
\item[append method] 92, 97, 101, 174, 175
\item[arc function] 31
\item[Archimedian spiral] 38
\item[argument] 17, 19, 21, 22, 26, 97
  \begin{itemize}
  \item gather 118
  \item keyword 33, 36, 191
  \item list 97
  \item optional 76, 79, 95, 107, 184
  \item positional 164, 169, 190
  \item variable-length tuple 118
  \end{itemize}
\item[argument scatter] 118
\item[arithmetic operator] 3
\item[assert statement] 159, 160
\item[assignment] 14, 63, 89
  \begin{itemize}
  \item augmented 93, 100
  \item item 74, 90, 116
  \item tuple 116, 117, 119, 122
  \end{itemize}
\item[assignment statement] 9
\item[attribute] 153, 169
  \begin{itemize}
  \item \_\_dict\_\_ 168
  \item class 172, 180
  \item initializing 168
  \item instance 148, 153, 172, 180
  \end{itemize}
\item[AttributeError] 152, 197
\item[augmented assignment] 93, 100
\item[Austen, Jane] 127
\item[average case] 202
\item[average cost] 208
\end{description}

\section{Ejemplos de código}

Aquí hay algunos ejemplos de código relevantes para términos del índice:

\begin{lstlisting}[language=Python]
# Ejemplo de función abs
x = -5
absolute_value = abs(x)
print(absolute_value)  # Output: 5

# Ejemplo de método append
my_list = [1, 2, 3]
my_list.append(4)
print(my_list)  # Output: [1, 2, 3, 4]

# Ejemplo de assert statement
def calculate_average(numbers):
    assert len(numbers) > 0, "List cannot be empty"
    return sum(numbers) / len(numbers)
\end{lstlisting}

\begin{lstlisting}[language=Python]
# Ejemplo de aliasing
a = [1, 2, 3]
b = a       # b es un alias de a
b[0] = 5    # Esto modifica tanto a como b
print(a)    # Output: [5, 2, 3]

# Para evitar aliasing
a = [1, 2, 3]
b = a.copy()  # b es una copia independiente
b[0] = 5     # Esto solo modifica b
print(a)     # Output: [1, 2, 3]
\end{lstlisting}

\section{Notas importantes}

El índice proporciona referencias cruzadas para todos los conceptos importantes cubiertos en el libro, incluyendo:

\begin{itemize}
\item Funciones incorporadas de Python
\item Métodos comunes
\item Conceptos de programación
\item Estructuras de datos
\item Técnicas de depuración
\item Diseño de algoritmos
\item Programación orientada a objetos
\end{itemize}

Para cada término, el número de página indica dónde encontrar la explicación detallada en el libro. Algunos términos tienen múltiples referencias cuando el concepto se discute en diferentes contextos a lo largo del texto.