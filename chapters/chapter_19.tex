\chapter{Las bondades del lenguaje}

Uno de mis objetivos para este libro ha sido enseñarte la menor cantidad de Python posible. Cuando existían dos maneras de hacer algo, elegí una y evité mencionar la otra. O, a veces, coloqué la segunda opción en un ejercicio.

Ahora quiero retomar algunas de las características interesantes que dejamos de lado. Python ofrece varias funciones que no son realmente necesarias —puedes escribir buen código sin ellas—, pero con ellas a veces puedes escribir código más conciso, legible o eficiente... y en ocasiones, todo a la vez.

\section{Expresiones condicionales}

Vimos sentencias condicionales en la Sección 5.4. Las sentencias condicionales suelen usarse para elegir uno de dos valores; por ejemplo:

\begin{lstlisting}[language=Python]
if x > 0:
    y = math.log(x)
else:
    y = float('nan')
\end{lstlisting}

Esta sentencia verifica si \(x\) es positivo. Si lo es, se calcula \texttt{math.log}. En caso contrario, \texttt{math.log} generaría un \texttt{ValueError}. Para evitar que el programa se detenga, generamos un "NaN", que es un valor especial de punto flotante que representa "No es un número".

Podemos escribir esta sentencia de manera más concisa usando una \textbf{expresión condicional}:

\begin{lstlisting}[language=Python]
y = math.log(x) if x > 0 else float('nan')
\end{lstlisting}

Esta línea se puede leer casi como en inglés: "y obtiene log(x) si x es mayor que 0; en caso contrario obtiene NaN".

A veces, las funciones recursivas pueden reescribirse utilizando expresiones condicionales. Por ejemplo, aquí hay una versión recursiva de factorial:

\begin{lstlisting}[language=Python]
def factorial(n):
    if n == 0:
        return 1
    else:
        return n * factorial(n-1)
\end{lstlisting}

Podemos reescribirla así:

\begin{lstlisting}[language=Python]
def factorial(n):
    return 1 if n == 0 else n * factorial(n-1)
\end{lstlisting}

Otro uso de las expresiones condicionales es manejar argumentos opcionales. Por ejemplo, este es el método \texttt{init} de \texttt{GoodKangaroo} (ver Ejercicio 17.2):

\begin{lstlisting}[language=Python]
def __init__(self, name, contents=None):
    self.name = name
    if contents == None:
        contents = []
    self.pouch_contents = contents
\end{lstlisting}

Podemos reescribirlo así:

\begin{lstlisting}[language=Python]
def __init__(self, name, contents=None):
    self.name = name
    self.pouch_contents = [] if contents == None else contents
\end{lstlisting}

En general, puedes reemplazar una sentencia condicional con una expresión condicional si ambas ramas contienen expresiones simples que se asignan o se retornan a la misma variable.

\section{Comprensiones de listas}

En la Sección 10.7 vimos los patrones \texttt{map} y \texttt{filter}. Por ejemplo, esta función toma una lista de cadenas, aplica el método \texttt{capitalize} a cada elemento y retorna una nueva lista:

\begin{lstlisting}[language=Python]
def capitalize_all(t):
    res = []
    for s in t:
        res.append(s.capitalize())
    return res
\end{lstlisting}

Podemos escribir esto de forma más concisa usando una comprensión de lista:

\begin{lstlisting}[language=Python]
def capitalize_all(t):
    return [s.capitalize() for s in t]
\end{lstlisting}

Los corchetes indican que estamos construyendo una nueva lista. La expresión dentro de los corchetes especifica los elementos, y la cláusula \texttt{for} indica sobre qué secuencia iteramos.

Las comprensiones de listas también se pueden usar para filtrar. Por ejemplo, esta función selecciona solo los elementos de \texttt{t} que están en mayúscula y devuelve una nueva lista:

\begin{lstlisting}[language=Python]
def only_upper(t):
    res = []
    for s in t:
        if s.isupper():
            res.append(s)
    return res
\end{lstlisting}

\section{Expresiones generadoras}

Podemos reescribirla usando una comprensión de lista:

\begin{lstlisting}[language=Python]
def only_upper(t):
    return [s for s in t if s.isupper()]
\end{lstlisting}

Las comprensiones de listas son concisas y fáciles de leer, al menos para expresiones simples. Y suelen ser más rápidas que los bucles \texttt{for} equivalentes, a veces mucho más.

Las \textbf{expresiones generadoras} son similares, pero usan paréntesis en lugar de corchetes:

\begin{lstlisting}[language=Python]
>>> g = (x**2 for x in range(5))
>>> g
<generator object <genexpr> at 0x7f4c45a786c0>
\end{lstlisting}

El resultado es un objeto generador que sabe cómo iterar por una secuencia de valores. Pero, a diferencia de las comprensiones de lista, no calcula todos los valores de una vez; espera a que se le pidan. La función \texttt{next} obtiene el siguiente valor del generador:

\begin{lstlisting}[language=Python]
>>> next(g)
0
>>> next(g)
1
\end{lstlisting}

Cuando se alcanza el final de la secuencia, \texttt{next} lanza una excepción \texttt{StopIteration}. También puedes usar un bucle \texttt{for}:

\begin{lstlisting}[language=Python]
>>> for val in g:
...     print(val)
4
9
16
\end{lstlisting}

Las expresiones generadoras suelen usarse con funciones como \texttt{sum}, \texttt{max} y \texttt{min}:

\begin{lstlisting}[language=Python]
>>> sum(x**2 for x in range(5))
30
\end{lstlisting}

\section{\texttt{any} y \texttt{all}}

Python provee la función incorporada \texttt{any}, que toma una secuencia de valores booleanos y devuelve \texttt{True} si alguno es verdadero. Funciona con listas:

\begin{lstlisting}[language=Python]
>>> any([False, False, True])
True
\end{lstlisting}

Pero frecuentemente se usa con expresiones generadoras:

\begin{lstlisting}[language=Python]
>>> any(letter == 't' for letter in 'monty')
True
\end{lstlisting}

Podríamos usar \texttt{any} para reescribir algunas funciones de búsqueda de la Sección 9.3. Por ejemplo, podríamos escribir \texttt{avoids} así:

\begin{lstlisting}[language=Python]
def avoids(word, forbidden):
    return not any(letter in forbidden for letter in word)
\end{lstlisting}

Python también ofrece la función \texttt{all}, que retorna \texttt{True} si todos los elementos de la secuencia son verdaderos.

\section{Conjuntos (\texttt{sets})}

En la Sección 13.6 usamos diccionarios para encontrar palabras que aparecen en un documento pero no en una lista de palabras. La función escrita tomaba \texttt{d1} con palabras del documento como claves, y \texttt{d2} con la lista de palabras. Retornaba un diccionario con las claves de \texttt{d1} que no estaban en \texttt{d2}.

\begin{lstlisting}[language=Python]
def subtract(d1, d2):
    res = dict()
    for key in d1:
        if key not in d2:
            res[key] = None
    return res
\end{lstlisting}

Python ofrece otro tipo incorporado llamado \texttt{set}, que se comporta como una colección de claves sin valores. Podemos reescribir \texttt{subtract} así:

\begin{lstlisting}[language=Python]
def subtract(d1, d2):
    return set(d1) - set(d2)
\end{lstlisting}

Algunos ejercicios del libro pueden resolverse de forma más concisa y eficiente usando conjuntos. Por ejemplo, esta solución a \texttt{has\_duplicates} (Ejercicio 10.7) usando diccionarios:

\begin{lstlisting}[language=Python]
def has_duplicates(t):
    d = {}
    for x in t:
        if x in d:
            return True
        d[x] = True
    return False
\end{lstlisting}

Se puede reescribir con conjuntos así:

\begin{lstlisting}[language=Python]
def has_duplicates(t):
    return len(set(t)) < len(t)
\end{lstlisting}

También podemos usar conjuntos en ejercicios del Capítulo 9. Por ejemplo, una versión de \texttt{uses\_only} con bucles:

\begin{lstlisting}[language=Python]
def uses_only(word, available):
    for letter in word:
        if letter not in available:
            return False
    return True
\end{lstlisting}

Reescrita con conjuntos:

\begin{lstlisting}[language=Python]
def uses_only(word, available):
    return set(word) <= set(available)
\end{lstlisting}

\section{\texttt{Counter}}

Un \texttt{Counter} es como un conjunto, pero si un elemento aparece más de una vez, lleva la cuenta de cuántas veces aparece. Se encuentra en el módulo estándar \texttt{collections}, así que se debe importar:

\begin{lstlisting}[language=Python]
>>> from collections import Counter
>>> count = Counter('parrot')
>>> count
Counter({'r': 2, 't': 1, 'o': 1, 'p': 1, 'a': 1})
\end{lstlisting}

Podemos usar \texttt{Counter} para reescribir \texttt{is\_anagram} del Ejercicio 10.6:

\begin{lstlisting}[language=Python]
def is_anagram(word1, word2):
    return Counter(word1) == Counter(word2)
\end{lstlisting}

\section{\texttt{defaultdict}}

El módulo \texttt{collections} también provee \texttt{defaultdict}, que es como un diccionario, pero si accedes a una clave inexistente, puede generar un nuevo valor automáticamente.

\begin{lstlisting}[language=Python]
>>> from collections import defaultdict
>>> d = defaultdict(list)
>>> t = d['new key']
>>> t
[]
\end{lstlisting}

Esto permite evitar la creación de listas innecesarias y simplificar el código, como en este ejemplo:

\begin{lstlisting}[language=Python]
def all_anagrams(filename):
    d = defaultdict(list)
    for line in open(filename):
        word = line.strip().lower()
        t = signature(word)
        d[t].append(word)
    return d
\end{lstlisting}

\section{Tuplas con nombre}

Muchos objetos simples son básicamente colecciones de valores relacionados. Python ofrece una forma más concisa de representarlos:

\begin{lstlisting}[language=Python]
from collections import namedtuple
Point = namedtuple('Point', ['x', 'y'])
\end{lstlisting}

Para crear un objeto \texttt{Point}, se usa la clase como función:

\begin{lstlisting}[language=Python]
>>> p = Point(1, 2)
>>> p
Point(x=1, y=2)
\end{lstlisting}

\section{Recolección de argumentos nombrados}

En la Sección 12.4 vimos cómo definir una función que recolecta sus argumentos en una tupla:

\begin{lstlisting}[language=Python]
def printall(*args):
    print(args)
\end{lstlisting}

Para recolectar argumentos nombrados, puedes usar el operador \texttt{**}:

\begin{lstlisting}[language=Python]
def printall(*args, **kwargs):
    print(args, kwargs)
\end{lstlisting}

\section{Ejercicios}

\subsection*{Ejercicio 19.1}

La siguiente es una función que calcula el coeficiente binomial de manera recursiva.

\begin{lstlisting}[language=Python]
def binomial_coeff(n, k):
    """Calcula el coeficiente binomial "n sobre k".

    n: número de ensayos
    k: número de éxitos

    retorna: int
    """
    if k == 0:
        return 1
    if n == 0:
        return 0

    res = binomial_coeff(n-1, k) + binomial_coeff(n-1, k-1)
    return res
\end{lstlisting}

Reescribe el cuerpo de la función usando expresiones condicionales anidadas.
